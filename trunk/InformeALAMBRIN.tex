\documentclass[a4paper,10pt]{article}

\usepackage[ansinew]{inputenc}
\usepackage[spanish]{babel}
\usepackage{graphicx}
\usepackage{listings}
\usepackage{appendix}
\usepackage{pdfpages}
\usepackage{fancyhdr}
\usepackage{ulem}
\pagestyle{fancy}

\def\dashuline{\bgroup 
  \ifdim\ULdepth=\maxdimen  % Set depth based on font, if not set already
   \settodepth\ULdepth{(j}\advance\ULdepth.4pt\fi
  \markoverwith{\kern.15em
  \vtop{\kern\ULdepth \hrule width .3em}%
  \kern.15em}\ULon}

\begin{document}

\lhead{\fancyplain{}{Base de Datos 75.15}}
\rhead{\fancyplain{}{Trabajo Pr\'actico Grupal}}

\setcounter{page}{2}

\newpage
\thispagestyle{empty}
\tableofcontents

\newpage
\section{Modelo Entidad Relaci\'on}
  \subsection{Diagrama}
    \begin{center}
      \includegraphics[width=18cm, height=12cm, angle=-90]{ModeloE-R/ModeloE-R.png}   
    \end{center}
  \subsection{Hip\'otesis}
    \begin{enumerate}
      \item Cada m\'ovil de exterior posee un id que lo identifica un\'ivocamente. 
      \item Un programa puede ser en vivo o grabado.
      \item S\'olo los programas en vivo poseen estudios de emisi\'on.
    \end{enumerate}
    
  \subsection{Diccionario}
    \subsubsection{Entidades}
    \begin{flushleft}
      \begin{large} \bf{Espacio} \end{large}
    \end{flushleft}
      \begin{tabular}{| p{2cm} | p{9cm} |}
	\hline
	\multicolumn{2}{|l|}{\bf{Descripci\'on:}} \\
	\hline
	\multicolumn{2}{|l|}{Un espacio es una divisi\'on del horario de transmisi\'on del canal.} \\
	\hline	
	\multicolumn{2}{|l|}{\bf{Especificaci\'on de atributos:}} \\
	\hline
	- D\'ia & El d\'ia en el que se emite el espacio. \\
	\hline \hline
	- Horario & El horario en el que se emite el espacio. \\
	\hline \hline
	- Duraci\'on & La duraci\'on del espacio.\\
	\hline \hline
	- Precio & El precio por segundo en el aire del espacio.\\
	\hline
	\multicolumn{2}{|l|}{\bf{Especificaci\'on de identificador \'unico:}} \\
	\hline
	\multicolumn{2}{|l|}{- D\'ia + Horario} \\
	\hline
      \end{tabular}
    
    \begin{flushleft}
      \begin{large} \bf{Tanda} \end{large}
    \end{flushleft}
      \begin{tabular}{| p{2cm} | p{9cm} |}
	\hline
	\multicolumn{2}{|l|}{\bf{Descripci\'on:}} \\
	\hline
	\multicolumn{2}{|l|}{Un tanda es un espacio donde se publicitan anuncios.} \\
	\hline	
	\multicolumn{2}{|l|}{\bf{Especificaci\'on de atributos:}} \\
	\hline
	- Hora\_tanda & La hora de comienzo de la tanda. \\
	\hline \hline
	- Nro\_tanda & El n\'umero de tanda correspondiente a un programa. \\
	\hline
	\multicolumn{2}{|l|}{\bf{Especificaci\'on de identificador \'unico:}} \\
	\hline
	\multicolumn{2}{|l|}{- D\'ia + Horario + Nro\_tanda} \\
	\hline
      \end{tabular}

    \begin{flushleft}
      \begin{large} \bf{Anuncio} \end{large}
    \end{flushleft}
      \begin{tabular}{| p{2cm} | p{9cm} |}
	\hline
	\multicolumn{2}{|l|}{\bf{Descripci\'on:}} \\
	\hline
	\multicolumn{2}{|l|}{Un anuncio es un espacio, dentro de una tanda, destinado a dar a conocer} \\
	\multicolumn{2}{|l|}{un producto.} \\
	\hline	
	\hline	
	\multicolumn{2}{|l|}{\bf{Especificaci\'on de atributos:}} \\
	\hline
	- T\'itulo & El t\'itulo del anuncio. \\
	\hline \hline
	- Duraci\'on & La duraci\'on del anuncio. \\
	\hline \hline
	- Producto \newline ofrecido & El producto ofrendido del anuncio. \\
	\hline
	\multicolumn{2}{|l|}{\bf{Especificaci\'on de identificador \'unico:}} \\
	\hline
	\multicolumn{2}{|l|}{- T\'itulo} \\
	\hline
      \end{tabular}

    \newpage
    \begin{flushleft}
      \begin{large} \bf{Anunciante} \end{large}
    \end{flushleft}
      \begin{tabular}{| p{2cm} | p{9cm} |}
	\hline
	\multicolumn{2}{|l|}{\bf{Descripci\'on:}} \\
	\hline
	\multicolumn{2}{|l|}{Un anunciante auspicia a lo sumo un producto, mediante un anuncio.} \\
	\hline	
	\multicolumn{2}{|l|}{\bf{Especificaci\'on de atributos:}} \\
	\hline
	- Nro\_cuenta & El n\'umero de cuenta del anunciante. \\
	\hline \hline
	- Domicilio & El domicilio del anunciante. \\
	\hline \hline
	- Raz\'on\_ \newline social & La raz\'on social del anunciante. \\
	\hline \hline
	- Tel\'efono & El tel\'efono del anunciate. \\
	\hline
	\multicolumn{2}{|l|}{\bf{Especificaci\'on de identificador \'unico:}} \\
	\hline
	\multicolumn{2}{|l|}{- Nro\_cuenta} \\
	\hline
      \end{tabular}

    \begin{flushleft}
      \begin{large} \bf{Programa} \end{large}
    \end{flushleft}
      \begin{tabular}{| p{2cm} | p{9cm} |}
	\hline
	\multicolumn{2}{|l|}{\bf{Descripci\'on:}} \\
	\hline
	\multicolumn{2}{|l|}{Un programa es un bloque con contenido que transmitir\'a el canal} \\
	\multicolumn{2}{|l|}{periodicamente} \\
	\hline	
	\multicolumn{2}{|l|}{\bf{Especificaci\'on de atributos:}} \\
	\hline
	- Clase & El tipo de programa que se transmite. \\
	\hline \hline
	- Nombre & Nombre del programa. \\
	\hline \hline
	- Duraci\'on & La duraci\'on del programa.\\
	\hline
	\multicolumn{2}{|l|}{\bf{Especificaci\'on de identificador \'unico:}} \\
	\hline
	\multicolumn{2}{|l|}{- Nombre} \\
	\hline
      \end{tabular}
  
    \begin{flushleft}
      \begin{large} \bf{Programa\_en\_vivo} \end{large}
    \end{flushleft}
      \begin{tabular}{| p{2cm} | p{9cm} |}
	\hline
	\multicolumn{2}{|l|}{\bf{Descripci\'on:}} \\
	\hline
	\multicolumn{2}{|l|}{Un programa en vivo es un programa que se transmite en vivo} \\
	\hline	
	\multicolumn{2}{|l|}{\bf{Especificaci\'on de atributos:}} \\
	\hline
	- No tiene & \\
	\hline
	\multicolumn{2}{|l|}{\bf{Especificaci\'on de identificador \'unico:}} \\
	\hline
	\multicolumn{2}{|l|}{- Hereda la clave de la entidad Programa} \\
	\hline
      \end{tabular} 
  
    \begin{flushleft}
      \begin{large} \bf{Programa\_grabado} \end{large}
    \end{flushleft}
      \begin{tabular}{| p{2cm} | p{9cm} |}
	\hline
	\multicolumn{2}{|l|}{\bf{Descripci\'on:}} \\
	\hline
	\multicolumn{2}{|l|}{Un programa grabado es un programa que se graba y luego se transmite} \\
	\hline	
	\multicolumn{2}{|l|}{\bf{Especificaci\'on de atributos:}} \\
	\hline
	- No tiene & \\
	\hline
	\multicolumn{2}{|l|}{\bf{Especificaci\'on de identificador \'unico:}} \\
	\hline
	\multicolumn{2}{|l|}{- Hereda la clave de la entidad Programa} \\
	\hline
      \end{tabular}

    \newpage
    \begin{flushleft}
      \begin{large} \bf{M\'ovil\_Exterior} \end{large}
    \end{flushleft}
      \begin{tabular}{| p{2cm} | p{9cm} |}
	\hline
	\multicolumn{2}{|l|}{\bf{Descripci\'on:}} \\
	\hline
	\multicolumn{2}{|l|}{Un m\'ovil exterior es un veh\'iculo preparado para usar en la calle durante} \\
	\multicolumn{2}{|l|}{la emisi\'on de programas en vivo} \\	
	\hline	
	\multicolumn{2}{|l|}{\bf{Especificaci\'on de atributos:}} \\
	\hline
	- Id\_movil & Identifica el movil. \\
	\hline \hline
	- Carac- \newline ter\'istica & Descripci\'on de las caracter\'isticas del movil. \\
	\hline
	\multicolumn{2}{|l|}{\bf{Especificaci\'on de identificador \'unico:}} \\
	\hline
	\multicolumn{2}{|l|}{- Id\_m\'ovil} \\
	\hline
      \end{tabular}
      
    \begin{flushleft}
      \begin{large} \bf{Estudio} \end{large}
    \end{flushleft}
      \begin{tabular}{| p{2cm} | p{9cm} |}
	\hline
	\multicolumn{2}{|l|}{\bf{Descripci\'on:}} \\
	\hline
	\multicolumn{2}{|l|}{Un estudio es el lugar donde se toma lugar un programa en vivo} \\
	\hline	
	\multicolumn{2}{|l|}{\bf{Especificaci\'on de atributos:}} \\
	\hline
	- Id\_estudio & Identifica el estudio. \\
	\hline \hline
	- Capacidad & Cantidad de personas que entran en el estudio. \\
	\hline
	\multicolumn{2}{|l|}{\bf{Especificaci\'on de identificador \'unico:}} \\
	\hline
	\multicolumn{2}{|l|}{- Id\_estudio} \\
	\hline
      \end{tabular} 
      
    \begin{flushleft}
      \begin{large} \bf{Productor\_independiente} \end{large}
    \end{flushleft}
      \begin{tabular}{| p{2cm} | p{9cm} |}
	\hline
	\multicolumn{2}{|l|}{\bf{Descripci\'on:}} \\
	\hline
	\multicolumn{2}{|l|}{Un productor independiente es aquel que se graba un programa y lo} \\
	\multicolumn{2}{|l|}{entrega al canal para ser transmitido} \\	
	\hline	
	\multicolumn{2}{|l|}{\bf{Especificaci\'on de atributos:}} \\
	\hline
	- Nro\_cuit & N\'umero de CUIT del productor. \\
	\hline \hline
	- Raz\'on\_ \newline social & Raz\'on social del productor\\
	\hline
	\multicolumn{2}{|l|}{\bf{Especificaci\'on de identificador \'unico:}} \\
	\hline
	\multicolumn{2}{|l|}{- Nro\_cuit} \\
	\hline
      \end{tabular} 
   
   
    \subsubsection{Interrelaciones}
    
    \begin{flushleft}
      \begin{large} \bf{Asignada\_a} \end{large}
    \end{flushleft}
      \begin{tabular}{| p{2cm} | p{9cm} |}
	\hline
	\multicolumn{2}{|l|}{\bf{Descripci\'on:}} \\
	\hline
	\multicolumn{2}{|l|}{Las tandas son asignadas a los espacios en una programaci\'on. Un espacio} \\
	\multicolumn{2}{|l|}{puede tener asignadas muchas tandas.} \\	
	\hline	
	\multicolumn{2}{|l|}{\bf{Especificaci\'on de identificador \'unico:}} \\
	\hline
	\multicolumn{2}{|l|}{- D\'ia + Horario + Nro\_tanda} \\
	\hline
      \end{tabular} 
   
    \begin{flushleft}
      \begin{large} \bf{Transmitido\_en} \end{large}
    \end{flushleft}
      \begin{tabular}{| p{2cm} | p{9cm} |}
	\hline
	\multicolumn{2}{|l|}{\bf{Descripci\'on:}} \\
	\hline
	\multicolumn{2}{|l|}{Los distintos programas son transmitidos en los diferentes espacios de la } \\
	\multicolumn{2}{|l|}{programaci\'on. Los programas pueden repetirse, por lo que pueden } \\	
	\multicolumn{2}{|l|}{tener m\'as de un espacio asignado.} \\	
	\hline	
	\multicolumn{2}{|l|}{\bf{Especificaci\'on de identificador \'unico:}} \\
	\hline
	\multicolumn{2}{|l|}{- Horario + D\'ia + Nombre} \\
	\hline
      \end{tabular}
       
    \begin{flushleft}
      \begin{large} \bf{Se\_publica} \end{large}
    \end{flushleft}
      \begin{tabular}{| p{2cm} | p{9cm} |}
	\hline
	\multicolumn{2}{|l|}{\bf{Descripci\'on:}} \\
	\hline
	\multicolumn{2}{|l|}{Durante las tandas son transmitidos los distintos anuncios(publicidad).} \\
	\multicolumn{2}{|l|}{Durante una tanda se pasan varios anuncios. Adem\'as, los anuncios } \\	
	\multicolumn{2}{|l|}{se repiten en varias tandas.} \\
	\hline
	\multicolumn{2}{|l|}{\bf{Especificaci\'on de atributos:}} \\
	\hline
	- Nro\_orden & Orden en que los anuncios se pasan durante una tanda. \\
	\hline
	\multicolumn{2}{|l|}{\bf{Especificaci\'on de identificador \'unico:}} \\
	\hline
	\multicolumn{2}{|l|}{- D\'ia + Horario + Nro\_tanda + Titulo} \\
	\hline
      \end{tabular}

    \begin{flushleft}
      \begin{large} \bf{Auspicia} \end{large}
    \end{flushleft}
      \begin{tabular}{| p{2cm} | p{9cm} |}
	\hline
	\multicolumn{2}{|l|}{\bf{Descripci\'on:}} \\
	\hline
	\multicolumn{2}{|l|}{Los anuncios son auspiciados por los anunciantes.} \\
	\hline	
	\multicolumn{2}{|l|}{\bf{Especificaci\'on de identificador \'unico:}} \\
	\hline
	\multicolumn{2}{|l|}{- Titulo + Nro\_cuenta} \\
	\hline
      \end{tabular}

    \begin{flushleft}
      \begin{large} \bf{Utiliza} \end{large}
    \end{flushleft}
      \begin{tabular}{| p{2cm} | p{9cm} |}
	\hline
	\multicolumn{2}{|l|}{\bf{Descripci\'on:}} \\
	\hline
	\multicolumn{2}{|l|}{Para realizar los programas en vivo, los mismos hacen uso de moviles para} \\
	\multicolumn{2}{|l|}{transmitirlos. Los mismos moviles son reutilizados por diferentes} \\	
	\multicolumn{2}{|l|}{programas.} \\
	\hline	
	\multicolumn{2}{|l|}{\bf{Especificaci\'on de identificador \'unico:}} \\
	\hline
	\multicolumn{2}{|l|}{- Id\_m\'ovil + Nombre} \\
	\hline
      \end{tabular}

    \begin{flushleft}
      \begin{large} \bf{Emitido\_en} \end{large}
    \end{flushleft}
      \begin{tabular}{| p{2cm} | p{9cm} |}
	\hline
	\multicolumn{2}{|l|}{\bf{Descripci\'on:}} \\
	\hline
	\multicolumn{2}{|l|}{Los programas en vivo se llevan a cabo en un estudio(puede que en m\'as de} \\
	\multicolumn{2}{|l|}{uno).} \\	
	\hline
	\multicolumn{2}{|l|}{\bf{Especificaci\'on de identificador \'unico:}} \\
	\hline
	\multicolumn{2}{|l|}{- Id\_estudio + Nombre} \\
	\hline
      \end{tabular} 
      
    \begin{flushleft}
      \begin{large} \bf{Provee} \end{large}
    \end{flushleft}
      \begin{tabular}{| p{2cm} | p{9cm} |}
	\hline
	\multicolumn{2}{|l|}{\bf{Descripci\'on:}} \\
	\hline
	\multicolumn{2}{|l|}{Los programas grabados que se emiten son provistos por productores} \\
	\multicolumn{2}{|l|}{ajenos al canal(productores independientes.} \\	
	\hline		
	\multicolumn{2}{|l|}{\bf{Especificaci\'on de identificador \'unico:}} \\
	\hline
	\multicolumn{2}{|l|}{- Nro\_cuit + Nombre} \\
	\hline
      \end{tabular}

\newpage
\section{Transformaci\'on del Modelo E-R a un Modelo Relacional}
  \subsection{Claves candidatas y for\'aneas}
  \begin{flushleft}
  Las claves candidatas est\'an separadas con punto y coma(;). \\
  En claves compuestas los atributos est\'an con punto y coma(;) y encerradas entre par\'entesis. \\
  Las claves for\'aneas son subrayadas con l\'inea punteada. 
  \end{flushleft}
    \begin{itemize}
      \item ESPACIO: (\underline{D\'ia}, \underline{Horario})
      \item PROGRAMA: \underline{Nombre}
      \item TANDA: (\dashuline{\underline{D\'ia}}, \dashuline{\underline{Horario}}, \underline{Nro\_tanda})
      \item SE\_PUBLICA: (\dashuline{\underline{D\'ia}}, \dashuline{\underline{Horario}}, \dashuline{\underline{Nro\_tanda}}, \dashuline{\underline{T\'itulo}})
      \item ANUNCIO: \underline{T\'itulo}
      \item ANUNCIANTE: \underline{Nro\_cuenta}; Raz\'on\_social
      \item PROGRAMA\_GRABADO: \underline{Nombre}
      \item PRODUCTOR\_INDEPENDIENTE: \underline{Nro\_cuit}; Raz\'on\_social
      \item PROGRAMA\_EN\_VIVO: \underline{Nombre}
      \item EMITIDO\_EN: (\dashuline{\underline{Nombre}}, \dashuline{\underline{Id\_estudio}})
      \item ESTUDIO: \underline{Id\_estudio}
      \item UTILIZA: \dashuline{(\underline{Nombre}}, \dashuline{\underline{Id\_m\'ovil}})
      \item MOVIL\_EXTERIOR: \underline{Id\_m\'ovil}
    \end{itemize}

  \subsection{Diagrama resultante de la transformaci\'on}
  \begin{flushleft}
    Las claves primarias son subrayadas con l\'inea s\'olida.
  \end{flushleft}

    \begin{flushleft}
      {\bf{ESPACIO}} (\underline{D\'ia}, \underline{Horario}, Duraci\'on, Precio, Nombre)
    \end{flushleft} 
 
    \begin{flushleft}
      {\bf{PROGRAMA}} (\underline{Nombre}, Duraci\'on, Clase)
    \end{flushleft} 

    \begin{flushleft}
      {\bf{TANDA}} (\underline{D\'ia}, \underline{Horario}, \underline{Nro\_tanda}, Hora\_tanda)
    \end{flushleft} 

    \begin{flushleft}
      {\bf{SE\_PUBLICA}} (\underline{D\'ia}, \underline{Horario}, \underline{Nro\_tanda}, \underline{T\'itulo}, Hora\_tanda, Nro\_orden)    
    \end{flushleft}

    \begin{flushleft}
      {\bf{ANUNCIO}} (\underline{T\'itulo}, Duraci\'on, Producto\_ofrecido, Nro\_cuenta)
    \end{flushleft}

    \begin{flushleft}
      {\bf{ANUNCIANTE}} (\underline{Nro\_cuenta}, Raz\'on\_social, Tel\'efono, Domicilio)
    \end{flushleft}

    \begin{flushleft}
      {\bf{PROGRAMA\_GRABADO}} (\underline{Nombre}, Nro\_cuit)
    \end{flushleft}
   
    \begin{flushleft}
      {\bf{PRODUCTOR\_INDEPENDIENTE}} (\underline{Nro\_cuit}, Raz\'on\_social)
    \end{flushleft}
  
    \begin{flushleft}
      {\bf{PROGRAMA\_EN\_VIVO}} (\underline{Nombre})
    \end{flushleft}

    \begin{flushleft}
      {\bf{EMITIDO\_EN}} (\underline{Nombre}, \underline{Id\_estudio})
    \end{flushleft}
  
    \begin{flushleft}
      {\bf{ESTUDIO}} (\underline{Id\_estudio}, Capacidad)
    \end{flushleft}
  
    \begin{flushleft}
      {\bf{UTILIZA}} (\underline{Nombre}, \underline{Id\_m\'ovil})
    \end{flushleft}
  
    \begin{flushleft}
      {\bf{MOVIL\_EXTERIOR}} (\underline{Id\_m\'ovil}, Caracter\'istica)
    \end{flushleft}

  \subsection{Atributos que pueden tomar valores nulos}
    \begin{flushleft}
      El Modelo Relacional tiene dos restricciones inherentes o impl\'icitas que refieren a las claves primarias y a las claves for\'aneas, llamadas reglas de integridad: 
    \end{flushleft}

    \begin{enumerate}
    \item REGLA DE INTEGRIDAD DE ENTIDAD: Los valores que forman parte de un clave primaria deben estar bien definidos.
    \item REGLA DE INTEGRIDAD REFERENCIAL: Los valores que toman una clave for\'anea deben ser o bien totalmente desconocidos o de lo contrario deben estar defenidos en la relaci\'on donde son clave primaria. 
    \end{enumerate}

    \begin{flushleft}
      A partir de esto \'ultimo, se puede definir que atributos pueden ser nulos.
    \end{flushleft}

    \begin{itemize}
      \item ESPACIO: Duraci\'on, Precio y Nombre.
      \item PROGRAMA: Duraci\'on y Clase.
      \item TANDA: Hora\_tanda.
      \item SE\_PUBLICA: Hora\_tanda y Nro\_orden.
      \item ANUNCIO: Duraci\'on, Producto\_ofrecido y Nro\_cuenta.
      \item ANUNCIANTE: Raz\'on\_social, Tel\'efono y Domicilio.
      \item PROGRAMA\_GRABADO: Nro\_cuit.
      \item PRODUCTOR\_INDEPENDIENTE: Raz\'on\_social.
      \item PROGRAMA\_EN\_VIVO: Ninguno.
      \item EMITIDO\_EN: Ninguno.
      \item ESTUDIO: Capacidad.
      \item UTILIZA: Ninguno.
      \item MOVIL\_EXTERIOR: Caracter\'istica.
    \end{itemize}
    
  \subsection{Diagrama de Modelos de tabla}
    \begin{flushleft}
    \end{flushleft}

%APENDICES
\appendix
\newpage
\section{Enunciado}
  \includepdf[pages=1-2, scale=0.8, pagecommand={\thispagestyle{plain}}]{EnunciadoBD.pdf}

\end{document}
