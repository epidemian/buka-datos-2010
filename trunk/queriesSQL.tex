\section{Consultas SQL}

% Configuración de paquete listings para SQL y tipografia courier.
\lstset{basicstyle=\fontfamily{pcr}\small, language=SQL}

\subsection*{1. \normalsize{Todos los programas que no se emiten los fines de semana.}}

\subsubsection*{Consulta}
\begin{lstlisting} 
select * from programa
where cod_prog not in (select cod_prog from espacio 
                       where dia in ('SAB', 'DOM'))
\end{lstlisting}

\subsubsection*{Resultado}
\begin{tabular}{|l|l|l|l|}
  \hline
    \bf{COD\_PROG} & \bf{NOMBRE} & \bf{CLASE} & \bf{DURACION} \\ 
  \hline
    C22 & Canal ``ETA'' & Humor & 1800 \\ 
  \hline
\end{tabular} 

\subsection*{2. \normalsize{El o los d\'ias de la semana en que el anunciante ``Juan Uncio'', anuncie su mayor cantidad diaria de anuncios.}}

\subsubsection*{Consulta}
\begin{lstlisting}
select dia 
from publicidad p, anuncio a, anunciante e
where p.titulo = a.titulo and 
      a.cuenta = e.cuenta and 
      razon_social = 'Juan Uncio'
group by dia
having count(1) >= all (select count(1)
                        from publicidad p, anuncio a, anunciante e
                        where p.titulo = a.titulo and 
                              a.cuenta = e.cuenta and
                              razon_social = 'Juan Uncio'
                        group by dia)
\end{lstlisting}

\subsubsection*{Resultado}
\begin{tabular}{|l|}
  \hline
    \bf{DIA} \\ 
  \hline
    MAR \\ 
  \hline
\end{tabular} 

\subsubsection*{Comentarios}

Las cl\'ausulas \lstinline|from|, \lstinline|where| y \lstinline|group by| son id\'enticas en ambos selects. En el select principal se obtienen los d\'ias en los que el anunciante ``Juan Uncio'' publica alg\'un anuncio, y en la cl\'ausula \lstinline|having| de \'este se compara la cantidad de anuncios por d\'ia contra su m\'aximo de anuncios por d\'ia. \\

La repetici\'on de c\'odigo (y la consiguiente dificultad para entenderlo/mantenerlo) podr\'ia evitarse usando una vista SQL:

\begin{lstlisting}
create view anuncios_por_dia_juan_uncio(dia, cantidad) as
select dia, count(1)
from publicidad p, anuncio a, anunciante e
where p.titulo = a.titulo and a.cuenta = e.cuenta and 
      razon_social = 'Juan Uncio'
group by dia;

select dia from anuncios_por_dia_juan_uncio
where cantidad = (select max(cantidad) from anuncios_por_dia_juan_uncio);
\end{lstlisting}

Pero esto, a su vez, pordr\'ia ser problem\'atico, ya que se estar\'ia creando una vista adicional en la base de datos que es de inter\'es s\'olo a una consulta particular. Adoptar esta metodolog\'ia para resolver todas las consultas complejas no ser\'ia una soluci\'on pr\'actica escalable, ya que pronto la base de datos se transformar\'ia en un conjunto de much\'isimas vistas poco reutilizables que, si bien no son costosas en t\'erminos de espacio f\'isico como las tablas, agregar\'ian complejidad innecesaria. \\

Lo ideal para no repetir el c\'odigo ser\'ia poder guardar una tabla como una variable temporal, que se descarte luego de salir del \'ambito en que se use. Desgraciadamente esta funcionalidad, m\'as procedural/imperativa, est\'a fuera del estandard SQL, que es m\'as bien un lenguaje declarativo. Algunas extensiones al lenguaje, como PSQL, proveen esta funcionalidad en forma de tablas temporales o variables locales a funciones.

\subsection*{3. \normalsize{La raz\'on social de los anunciantes que anuncian en la mayor cantidad de programas distintos.}}

\subsubsection*{Consulta}
\begin{lstlisting} 
select a.razon_social
from anunciante a, anuncio an, publicidad p, espacio e
where a.cuenta = an.cuenta and an.titulo = p.titulo and 
      p.dia = e.dia and p.hora = e.hora
group by a.razon_social
having count(distinct e.cod_prog) >= all 
       (select count(distinct e.cod_prog)
        from anunciante a, anuncio an, publicidad p, espacio e
        where a.cuenta = an.cuenta and an.titulo = p.titulo and 
              p.dia = e.dia and p.hora = e.hora
        group by a.razon_social)
\end{lstlisting}

\subsubsection*{Resultado}
\begin{tabular}{|l|}
  \hline
    \bf{RAZON\_SOCIAL} \\ 
  \hline
    Ricardo Perez \\
    Juan Uncio \\ 
  \hline
\end{tabular} 

\subsubsection*{Comentarios}
Al igual que en la consulta 2, en \'este ejemplo tambi\'en se podr\'ia evitar repetici\'on de c\'odigo usando una vista SQL:

\begin{lstlisting} 
create view programas_por_anunciante(razon_social, programas) as
select razon_social, count(distinct e.cod_prog)
from anunciante a, anuncio an, publicidad p, espacio e
where a.cuenta = an.cuenta and an.titulo = p.titulo and 
      p.dia = e.dia and p.hora = e.hora
group by a.razon_social;

select razon_social from programas_por_anunciante
where programas = (select max(programas) from programas_por_anunciante);
\end{lstlisting}

\subsection*{4. \normalsize{El nombre de los programas grabados que se emitan en promedio m\'as veces por d\'ia los fines de semana que los d\'ias laborales (lunes a viernes).}}

\subsubsection*{Consulta}
\begin{lstlisting} 
select nombre from programa p
where cod_prog in (select cod_prog from produccion) and
      (select count(*) from espacio
       where cod_prog = p.cod_prog and dia in ('SAB', 'DOM')) / 2 
         >
      (select count(*) from espacio
       where cod_prog = p.cod_prog and dia not in ('SAB', 'DOM')) / 5
\end{lstlisting}

\subsubsection*{Resultado}
\begin{tabular}{|l|}
  \hline
    \bf{NOMBRE} \\ 
  \hline
    La Aventura del Punto \\ 
  \hline
\end{tabular} 


\subsubsection*{Suposici\'on}
Los programas grabados son los que tienen una producci\'on asociada. \\

\subsubsection*{Comentarios}
En el primer subquery se cuenta la cantidad de veces que se transmite el programa los fines de semana, y el segundo la cantidad de veces que se transmite los d\'ias de semana. El promedio se saca haciendo una divisi\'on por la cantidad de d\'ias en cada caso, y luego s\'olo se comparan los promedios.


\subsection*{5. \normalsize{El d\'ia, hora y n\'umero de tanda de las tandas que tengan una cantidad de anuncios mayor que el promedio de anuncios por tanda del mismo d\'ia.}}

\subsubsection*{Consulta}
\begin{lstlisting} 
select dia, hora, tanda
from publicidad p
group by dia, hora, tanda
having count(distinct titulo) > (select avg(count(distinct titulo))
                                 from publicidad
                                 where dia = p.dia
                                 group by dia, hora, tanda)
\end{lstlisting}

O bien, calculando el promedio \textit{a mano}:
\begin{lstlisting}
select dia, hora, tanda
from publicidad p
group by dia, hora, tanda
having count(distinct titulo) > (select count(1) from publicidad 
                                 where dia = p.dia) /
                                (select count(1) from tanda
                                 where dia = p.dia)
\end{lstlisting}

O bien, seleccionando directamente de la tabla \lstinline|tanda|:
\begin{lstlisting}
select * from tanda t
where (select count (distinct titulo) from publicidad 
       where dia = t.dia and hora = t.hora and tanda = t.tanda) 
         > 
      (select count(1) from publicidad where dia = t.dia) /
      (select count(1) from tanda where dia = t.dia)
\end{lstlisting}

\subsubsection*{Resultado}
\begin{tabular}{|l|l|l|}
  \hline
    \bf{DIA} & \bf{HORA} & \bf{TANDA} \\ 
  \hline
    DOM & 13:00 & 1 \\
    DOM & 13:00 & 2 \\
    SAB & 11:00 & 1 \\
    SAB & 13:00 & 2 \\
  \hline
\end{tabular} 

\subsubsection*{Suposici\'on}
El enunciado dice ``...las tandas que tengan una cantidad de anuncios...'', con lo cual se supone que lo que se desea tener en cuenta es la cantidad de anuncios distintos en esa tanda, y no la cantidad de publicidades. e.g: si una tanda tiene 3 publicidades, pero las 3 son del mismo anuncio (un lavaje de cerebro importante), se contaría un sólo anuncio en esa tanda. \\

Si lo que se quería era contar la cantidad de publicidades en una tanda, sin importar que se tratara de anuncios repetidos, basta con reemplazar los ``\lstinline|distinct titulo|'' por \lstinline|*| o \lstinline|1| (en cualquiera de las consultas).

\subsubsection*{Comentarios}
Esta consulta es un ejemplo de que ocurre a menudo con SQL: existen muchas formas de obtener la misma información. La elección de la consulta podrá depender de lo que el programador considere más expresivo, o podrían llevarse a cabo pruebas de performance para averiguar cuál es la más conveniente. \\

En este caso, se considera que la forma más expresiva de calcular el promedio de anuncios por tanda de un día dado es la usada en la primer consulta. Pero es posible que la expresión ``\lstinline|avg(count(distinct titulo))|'' no sea soportada por todos los motores de bases de datos, y se tenga que recurrir a calcular el promedio \textit{a mano} como en la segunda o tercer consulta.


\subsection*{6. \normalsize{Para cada d\'ia, mostrar el d\'ia, la hora y el nombre del programa que se emite en el espacio con la mayor cantidad de anuncios distintos de cada d\'ia.}}

\subsubsection*{Consulta}
\begin{lstlisting} 
select e.dia, e.hora, nombre
from espacio e, programa pr, publicidad pu
where e.cod_prog = pr.cod_prog and 
      e.dia = pu.dia and e.hora = pu.hora
group by e.dia, e.hora, nombre
having count(distinct titulo) >= all (select count(distinct titulo) 
                                      from publicidad 
                                      where dia = e.dia 
                                      group by dia, hora)
order by e.dia, e.hora
\end{lstlisting}

\subsubsection*{Resultado}
\begin{tabular}{|l|l|l|}
  \hline
    \bf{DIA} & \bf{HORA} & \bf{NOMBRE} \\ 
  \hline
    DOM & 13:00 & Su Ojo en la Lupa \\ 
    JUE & 11:00 & Viva la Patria \\ 
    JUE & 13:00 & Canal ``ETA'' \\ 
    JUE & 15:00 & Viva la Patria \\ 
    LUN & 11:00 & La Aventura del Punto \\ 
    LUN & 13:00 & Canal ``ETA'' \\ 
    LUN & 15:00 & Viva la Patria \\ 
    MAR & 11:00 & La Aventura del Punto \\ 
    MAR & 13:00 & Canal ``ETA'' \\ 
    MAR & 15:00 & Viva la Patria \\ 
    MIE & 11:00 & La Aventura del Punto \\ 
    MIE & 13:00 & Canal ``ETA'' \\ 
    MIE & 15:00 & Viva la Patria \\ 
    SAB & 13:00 & Su Ojo en la Lupa \\ 
    VIE & 11:00 & Viva la Patria \\ 
    VIE & 13:00 & Su Ojo en la Lupa \\ 
    VIE & 15:00 & Viva la Patria \\ 
  \hline
\end{tabular} 

\subsubsection*{Comentarios}
Como puede existir más de un espacio con la mayor cantidad de anuncios distintos por día, aparecen múltiples resultados para algunos de los días de la semana.

\subsection*{7. \normalsize{El nombre de los programas en vivo que se emiten en todos los d\'ias h\'abiles (lunes a viernes).}}

\subsubsection*{Consulta}

\begin{lstlisting} 
select nombre 
from programa
where cod_prog not in (select cod_prog from produccion) and
      cod_prog in (select cod_prog from espacio where dia = 'LUN') and
      cod_prog in (select cod_prog from espacio where dia = 'MAR') and
      cod_prog in (select cod_prog from espacio where dia = 'MIE') and
      cod_prog in (select cod_prog from espacio where dia = 'JUE') and
      cod_prog in (select cod_prog from espacio where dia = 'VIE')
\end{lstlisting}

\subsubsection*{Resultado}
\begin{tabular}{|l|}
  \hline
    \bf{NOMBRE} \\ 
  \hline
    Su Ojo en la Lupa \\ 
  \hline
\end{tabular} 

\subsubsection*{Suposición}
Los programas en vivo son los no grabados, es decir, los que no tienen una producción asociada. \\

\subsubsection*{Comentarios}
Esta consulta está expresada como una especie de interesección: \textit{``Los programas en vivo que se transmiten los Lunes'' $\cap$ \ldots $\cap$``Los programas en vivo que se transmiten los Viernes''}. \\

Pero también podría verse como una división del álgebra relacional:
\begin{lstlisting} 
select nombre 
from programa p
where cod_prog not in (select cod_prog from produccion) and
      not exists (select * 
                  from (values 'LUN', 'MAR', 'MIE', 'JUE', 'VIE') as habiles
                    minus
                  select dia from espacio 
                  where cod_prog = p.cod_prog)
\end{lstlisting}
 
La cual puede \textit{leerse} como \textit{``Los programas en vivo tal que no exista día hábil en que el programa no se transmita''}. lo cual resulta bastante más natural y sencillo de expresar que la consulta con interesección anterior usada. Por desgracia, aunque ésta consulta es en teoría válida\cite{values_expression}, el gestor de base de datos que utiliza el sistema de pruebas no permite utilizar una expresión \lstinline|values| para crear una tabla constante. Y no existe forma de obtener los valores de los días de la semana ya que estos no estás cargados en ninguna tabla, y tomar la suposición de que están al menos una vez en la talba \texttt{espacio} sería algo muy poco \textit{robusto}. \\

\subsection*{8. \normalsize{Los d\'ias en que se ocupan todos los estudios.}}

\subsubsection*{Consulta}
\begin{lstlisting} 
select distinct dia from espacio e
where not exists (select estudio from estudio
                    minus
                  select estudio 
                  from espacio es, ocupacion o
                  where es.cod_prog = o.cod_prog and es.dia = e.dia)
\end{lstlisting}

\subsubsection*{Resultado}
\begin{tabular}{|l|}
  \hline
    \bf{DIA} \\ 
  \hline
    JUE \\
    LUN \\
    MAR \\
    MIE \\
  \hline
\end{tabular} 

\subsection*{9. \normalsize{La raz\'on social de los anunciantes que tienen al menos un anuncio de sus productos en las emisiones de todos los programas del productor ``ROMUALDO''.}}

La especificación de la consulta resulta ambigua. Existen, a saber, dos interprestaciones posibles, para las cuales se exponen sus respectivas soluciones.

\subsubsection*{Consulta (1)}

\begin{lstlisting} 
select razon_social
from anunciante a
where not exists (
    -- Los espacios de los programas de 'Romualdo'.
    select 1
    from productor pdt, produccion pdc, espacio e
    where razon_social = 'Romualdo' and
          pdt.cuit = pdc.cuit and 
          pdc.cod_prog = e.cod_prog and
          not exists (-- Las publicidades de 'a' en ese espacio.
                      select 1 
                      from publicidad pub, anuncio an
                      where dia = e.dia and hora = e.hora and
                            pub.titulo = an.titulo and
                            an.cuenta = a.cuenta ))
\end{lstlisting}

\subsubsection*{Resultado (1)}
\texttt{no data found}

\subsubsection*{Comentarios (1)}
En esta consulta se buscan a los anunciantes que tienen al menos un anuncio en todas las emisiones de todos los programas de ``Romualdo''. La consulta está expresada como \textit{``Los anunciantes tales que no existe emisión de ``Romualdo'' que no tenga algúna publicidad de alguno de sus productos''} \\

El resultado vacío de la consulta se debe a que no hay ningún anunciante con anuncios en todas las emisiones de ``Romualdo''. En particular, hay 14 emisiones de programas de ``Romualdo'' a la semana, y ningún anunciante llega a esa cantidad de publicidades entre todos sus productos. Con lo cual, resulta evidente que no existe anunciante alguno que cumpla con la condición pedida. \\

\subsubsection*{Consulta (2)}

\begin{lstlisting} 
select razon_social
from anunciante a
where not exists (
    -- Los programas de 'Romualdo'.
    select 1
    from productor pdt, produccion pdc
    where razon_social = 'Romualdo' and
          pdt.cuit = pdc.cuit and 
          not exists (-- Los espacios donde se publicita algun producto de 'a'
                      select 1 
                      from espacio e, publicidad pub, anuncio an
                      where e.cod_prog = pdc.cod_prog and
                            e.dia = pub.dia and e.hora = pub.hora and
                            pub.titulo = an.titulo and
                            an.cuenta = a.cuenta ))
\end{lstlisting}

\subsubsection*{Resultado (2)}
\begin{tabular}{|l|}
  \hline
    \bf{RAZON\_SOCIAL} \\ 
  \hline
    Ricardo Perez \\ 
    Juan Uncio \\
  \hline
\end{tabular} 

\subsubsection*{Comentarios (2)}
En esta otra consulta se buscan los anunciantes que tienen al menos un anuncio en alguna emisión de todos los programas de ``Romualdo''. Ésta se puede leer como \textit{``Los anunciantes tales que no existe programa de ``Romualdo'' que no tenga alguna publicidad de alguno de los productos del anunciante en alguna de sus emisiones''}. La diferencia fundamental con la consulta anterior es que la condición se reduce sólo a todos los programas, que en el caso de ``Romualdo'' son sólo 2. Los anunciantes que tengan algún anuncio en alguna emisión de ambos programas ya cumplirán con la condición. \\

\subsection*{10. \normalsize{El nombre de los programas que se emiten en todos los d\'ias en que al menos una vez sean utilizados todos los m\'oviles por cualquiera de los programas emitidos en ese d\'ia.}}

\subsubsection*{Consulta}
\begin{lstlisting} 

\end{lstlisting}

\subsubsection*{Resultado}
\begin{tabular}{|l|l|}
  \hline
    \bf{CAMPO1} & \bf{CAMPO2} \\ 
  \hline
    c11 & c21 \\ 
    c21 & c22 \\
  \hline
\end{tabular} 
