\section{Consultas SQL}

% Configuración de 
\lstset{basicstyle=\fontfamily{pcr}\small, language=SQL}

\subsection*{1. \normalsize{Todos los programas que no se emiten los fines de semana.}}

\subsubsection*{Consulta}
\begin{lstlisting} 
select * from programa
  minus
select p.* from programa p, espacio e
where p.cod_prog = e.cod_prog and e.dia in ('SAB', 'DOM')
\end{lstlisting}

O bien:

\begin{lstlisting} 
select * from programa p
where cod_prog not in (select e.cod_prog from espacio e 
                       where e.cod_prog = p.cod_prog and 
                             dia in ('SAB', 'DOM'))
\end{lstlisting}

\subsubsection*{Resultados}
\begin{tabular}{|l|l|l|l|}
  \hline
    \bf{COD\_PROG} & \bf{NOMBRE} & \bf{CLASE} & \bf{DURACION} \\ 
  \hline
    C22 & Canal ``ETA'' & Humor & 1800 \\ 
  \hline
\end{tabular} 

\subsection*{2. \normalsize{El o los d\'ias de la semana en que el anunciante ``Juan Uncio'', anuncie su mayor cantidad diaria de anuncios.}}

\subsubsection*{Consulta}
\begin{lstlisting}
select dia 
from publicidad p, anuncio a, anunciante e
where p.titulo = a.titulo and 
      a.cuenta = e.cuenta and 
      razon_social = 'Juan Uncio'
group by dia
having count(1) = (select max(count(1))
                   from publicidad p, anuncio a, anunciante e
                   where p.titulo = a.titulo and 
                         a.cuenta = e.cuenta and
                         razon_social = 'Juan Uncio'
                   group by dia)
\end{lstlisting}

\subsubsection*{Resultados}
\begin{tabular}{|l|}
  \hline
    \bf{DIA} \\ 
  \hline
    MAR \\ 
  \hline
\end{tabular} 

\subsubsection*{Comentarios}

Las cl\'ausulas \verb|from|, \verb|where| y \verb|group by| son id\'enticas en ambos selects. En el select principal se obtienen los d\'ias en los que el anunciante ``Juan Uncio'' publica alg\'un anuncio, y en la cl\'ausula \verb|having| de \'este se compara la cantidad de anuncios por d\'ia contra su m\'aximo de anuncios por d\'ia. \\

La repetici\'on de c\'odigo podr\'ia evitarse usando una vista SQL:

\begin{lstlisting}
create view anuncios_por_dia_juan_uncio(dia, cantidad) as
select dia, count(1)
from publicidad p, anuncio a, anunciante e
where p.titulo = a.titulo and a.cuenta = e.cuenta and 
      razon_social = 'Juan Uncio'
group by dia;

select dia from anuncios_por_dia_juan_uncio
where cantidad = (select max(cantidad) from anuncios_por_dia_juan_uncio);
\end{lstlisting}


\subsection*{3. \normalsize{La raz\'on social de los anunciantes que anuncian en la mayor cantidad de programas distintos.}}

\subsubsection*{Consulta}
\begin{lstlisting} 
select a.razon_social
from anunciante a, anuncio an, publicidad p, espacio e
where a.cuenta = an.cuenta and an.titulo = p.titulo and 
      p.dia = e.dia and p.hora = e.hora
group by a.razon_social
having count(distinct e.cod_prog) = 
       (select max(count(distinct e.cod_prog))
        from anunciante a, anuncio an, publicidad p, espacio e
        where a.cuenta = an.cuenta and an.titulo = p.titulo and 
              p.dia = e.dia and p.hora = e.hora
        group by a.razon_social)
\end{lstlisting}

\subsubsection*{Resultados}
\begin{tabular}{|l|}
  \hline
    \bf{RAZON\_SOCIAL} \\ 
  \hline
    Juan Uncio \\ 
  \hline
\end{tabular} 

\subsubsection*{Comentarios}
En \'este ejemplo tambi\'en se podr\'ia evitar repetici\'on de c\'odigo usando una vista SQL:

\begin{lstlisting} 
create view programas_por_anunciante(razon_social, programas) as
select razon_social, count(distinct e.cod_prog)
from anunciante a, anuncio an, publicidad p, espacio e
where a.cuenta = an.cuenta and an.titulo = p.titulo and 
      p.dia = e.dia and p.hora = e.hora
group by a.razon_social;

select razon_social from programas_por_anunciante
where programas = (select max(programas) from programas_por_anunciante);

\end{lstlisting}

\subsection*{4. \normalsize{El nombre de los programas grabados que se emitan en promedio m\'as veces por d\'ia los fines de semana que los d\'ias laborales (lunes a viernes).}}

\subsubsection*{Consulta}
\begin{lstlisting} 
select p.nombre from programa p
where p.cod_prog in (select cod_prog from produccion) and
      (select count(*) from espacio e 
       where e.cod_prog = p.cod_prog and e.dia in ('SAB', 'DOM')) / 2 
         >
      (select count(*) from espacio e 
       where e.cod_prog = p.cod_prog and e.dia not in ('SAB', 'DOM')) / 5
\end{lstlisting}

\subsubsection*{Resultados}
\begin{tabular}{|l|}
  \hline
    \bf{NOMBRE} \\ 
  \hline
    La Aventura del Punto \\ 
  \hline
\end{tabular} 

\subsubsection*{Comentarios}
Suposici\'on: Los programas grabados son los que tienen una producci\'on asociada. \\

En el primer subquery se cuenta la cantidad de veces que se transmite el programa los fines de semana, y el segundo la cantidad de veces que se transmite los d\'ias de semana. El promedio se saca haciendo una divisi\'on por la cantidad de d\'ias en cada caso, y luego s\'olo se comparan los promedios.


\subsection*{5. \normalsize{El d\'ia, hora y n\'umero de tanda de las tandas que tengan una cantidad de anuncios mayor que el promedio de anuncios por tanda del mismo d\'ia.}}

\subsubsection*{Consulta}
\begin{lstlisting} 

\end{lstlisting}

\subsubsection*{Resultados}
\begin{tabular}{|l|l|}
  \hline
    \bf{CAMPO1} & \bf{CAMPO2} \\ 
  \hline
    c11 & c21 \\ 
    c21 & c22 \\
  \hline
\end{tabular} 

\subsection*{6. \normalsize{Para cada d\'ia, mostrar el d\'ia, la hora y el nombre del programa que se emite en el espacio con la mayor cantidad de anuncios distintos de cada d\'ia.}}

\subsubsection*{Consulta}
\begin{lstlisting} 

\end{lstlisting}

\subsubsection*{Resultados}
\begin{tabular}{|l|l|}
  \hline
    \bf{CAMPO1} & \bf{CAMPO2} \\ 
  \hline
    c11 & c21 \\ 
    c21 & c22 \\
  \hline
\end{tabular} 

\subsection*{7. \normalsize{El nombre de los programas en vivo que se emiten en todos los d\'ias h\'abiles (lunes a viernes).}}

\subsubsection*{Consulta}
\begin{lstlisting} 

\end{lstlisting}

\subsubsection*{Resultados}
\begin{tabular}{|l|l|}
  \hline
    \bf{CAMPO1} & \bf{CAMPO2} \\ 
  \hline
    c11 & c21 \\ 
    c21 & c22 \\
  \hline
\end{tabular} 

\subsection*{8. \normalsize{Los d\'ias en que se ocupan todos los estudios.}}

\subsubsection*{Consulta}
\begin{lstlisting} 

\end{lstlisting}

\subsubsection*{Resultados}
\begin{tabular}{|l|l|}
  \hline
    \bf{CAMPO1} & \bf{CAMPO2} \\ 
  \hline
    c11 & c21 \\ 
    c21 & c22 \\
  \hline
\end{tabular} 

\subsection*{9. \normalsize{La raz\'on social de los anunciantes que tienen al menos un anuncio de sus productos en las emisiones de todos los programas del productor ``ROMUALDO''.}}

\subsubsection*{Consulta}
\begin{lstlisting} 

\end{lstlisting}

\subsubsection*{Resultados}
\begin{tabular}{|l|l|}
  \hline
    \bf{CAMPO1} & \bf{CAMPO2} \\ 
  \hline
    c11 & c21 \\ 
    c21 & c22 \\
  \hline
\end{tabular} 

\subsection*{10. \normalsize{El nombre de los programas que se emiten en todos los d\'ias en que al menos una vez sean utilizados todos los m\'oviles por cualquiera de los programas emitidos en ese d\'ia.}}

\subsubsection*{Consulta}
\begin{lstlisting} 

\end{lstlisting}

\subsubsection*{Resultados}
\begin{tabular}{|l|l|}
  \hline
    \bf{CAMPO1} & \bf{CAMPO2} \\ 
  \hline
    c11 & c21 \\ 
    c21 & c22 \\
  \hline
\end{tabular} 
